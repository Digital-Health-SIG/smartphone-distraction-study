\documentclass[ngerman]{tutorial}

\usepackage{babel}
\usepackage{array}
\usepackage{graphicx}
\usepackage{csquotes} \let\say\enquote
\usepackage{fontawesome}

% customs
\usepackage{tikz-qr,tikz-mobile}
\def\mobiledefaultwidth{\marginparwidth}
% \appto\sidenoteformat{\def\mobiledefaultwidth{\marginparwidth}}
\usetikzlibrary{arrows.meta}

\title{App-Tutorial}
\author{Anton Vlasjuk}
\date{2022}

\titleheight{6.65cm}% aktuell noch zum rumprobieren
\logopadding{1.65cm}% padding um uulm -- gilt nur für oben rechts
\titleimage{corner_1}% mit \titleimage[8cm] kann man eine Manuelle Größe setzen

\appto\captionsngerman{\appto\contentsname{\footnote{Bei weiteren Fragen melde dich gerne bei \mail[Anton Vlasjuk]{anton.vlasjuk@uni-ulm.de}.}}\def\figurename{Abb.}\def\tablename{Tbl.}}

\setcounter{tocdepth}{1}

\graphicspath{{figures}}

\begin{document}

\maketitle
\tableofcontents
\bigskip\vfill

\section{Installieren}
Scanne folgenden QR-Code und installiere die App als \texttt{.apk}-Datei:\footnote{Ein Tutorial zur Installation einer APK findet sich hier \href{https://www.heise.de/tipps-tricks/Externe-Apps-APK-Dateien-bei-Android-installieren-so-klappt-s-3714330.html}{per Klick}.}
\vfill
\begin{center}
    \colorlet{qr@fancy@gradient@tl}{black!98!@page}
    \colorlet{qr@fancy@gradient@br}{black!98!@page!80!@main}
    \hidelink{\fancyqr[height=6cm]{https://drive.google.com/file/d/1hOs30yUwAKuoD-zYpWYmDVbber9gCrsO/view?usp=sharing}}
\end{center}
\vfill
\clearpage

\section{Berechtigungen}
\sidefig[start-per]{\mobile{start_per.png}}{Die App fragt nach Berechtigungen.}%
Beim Starten der App wird nach App-Berechtigungen gefragt, welche für die Funktionalität der App notwendig sind.
Bitte \button{Ja} auswählen, dann taucht eine Liste der installierten Apps auf. Hier soll dann entsprechend die \say{Lernstudie} ausgewählt und zuletzt die Erlaubnis erteilt werden:
\begin{center}
    \mobile{list_per.png}\quad\tikz{\draw[-Kite,rounded corners=1pt] (0,0) -- (.5,0) |- (1,.75);}\quad\mobile{item_per.png}
\end{center}

Falls \button{Nein} oder \button{Nicht Erinnern} am Anfang ausgewählt wurde, so kann man die Berechtigungen nachträglich in den Einstellungen geben (Zahnrad in \autoref{start-per}):
\begin{center}
    \mobile{settings_per.png}
\end{center}
Einfach entsprechend auf die Felder klicken.



\clearpage
\section{Anmelden}
\sidefig[start-reg]{\mobile{start_reg.png}}{Die Anmeldung}
Gebe einen Namen ein, der 3--6 Zeichen lang ist und nur Buchstaben (a--z) oder Zahlen enthält (0--9). Beispielsweise \say{test11}. Gebe zusätzlich dein Geschlecht und deinen Studiengang an. Wenn du nun eine stabile Internetverbindung hast, dann drücke auf \button{Anmelden} (bei Erfolg ist keine Änderung mehr möglich; bei einem Fehlschlag kommt eine entsprechende Meldung).
\begin{center}
    \tabular{c@{\hskip4em}c}
        Fehlschlag: & Erfolg: \\
        \mobile{reg_fail.png} & \mobile{reg_succ.png}
    \endtabular
\end{center}



\section{Weitere Features im Hauptmenü}
\sidefig[options]{\mobile{options.png}}{Das Hauptmenü inklusive Optionen.}Es gibt drei weitere Features im Hauptmenü. Klicke hierfür auf den runden Button~\icon{options_menu} unten rechts in \autoref{start-reg} und betrachte \autoref{options}:
\begin{description}
    \item[\icon{options_update}] Als erstes kann man seine Daten aktualisieren (Daten an meinen Server schicken), falls mal keine Internetverbindung da war oder ähnliches.
    \item[\icon{options_mail}] Als nächstes kann man eine E-Mail an mich schicken, falls die App nicht funktionieren sollte oder andere Probleme bestehen.
    \item[\icon{options_token}] Zuletzt kann man sein Token erneuern, falls oft Meldungen kommen, dass man nicht angemeldet oder das Token ungültig sei.
    Falls dies wirklich nötig ist, so schreibe mir bitte eine E-Mail, dann können wir gemeinsam das Token erneuern (siehe \autoref{token}).
\end{description}

\clearpage
\sidefig[token]{\mobile{token_new.png}}{Ein neues Token anfordern.}
\section{Lernsitzung}
Um auf die Lernsitzung zu kommen, musst du oben links auf das Menü~(\faNavicon) klicken (oder von links nach rechts wischen). Dann öffnet sich folgendes Menü, bei dem auf \say{Lernsitzung} geklickt werden soll:
\begin{center}
    \mobile{options.png}\quad\tikz{\draw[-Kite,rounded corners=1pt] (0,0) -- (.5,0) |- (1,.75);}\quad\mobile{nav_learn.png}
\end{center}

Nun befinden wir uns am Anfang der Lernsitzung (\autoref{lernsitzung}).
\sidefig[lernsitzung]{\mobile{learn_start.png}}{Der Anfang einer Lernsitzung.}
Hier kannst du einstellen, wie lang deine Lernsitzung gehen soll: Dauer einer Lernphase, Pause und die Anzahl der Wiederholungen (die Gesamtanzahl an Lernphasen). Wenn man mit der Einstellung zufrieden ist, kann man mit einem Klick auf \button{Start} die Lernphase beginnen.

\subsection{Dauerhafte Konfiguration}
Zuvor aber noch eine Kleinigkeit: Falls man die Einstellung nicht immer wieder neu einrichten möchte, so kannst du bei den Einstellungen (Zahnrad oben rechts) einen Standard festlegen:
\begin{center}
    \mobile{settings_learn.png}\qquad\mobile{settings_alarm.png}
\end{center}
Hier kann man auch den Alarm regulieren (Dauer und Lautstärke)~--- wenn eine Phase auf natürliche Weise endet, startet ein Alarm.

\subsection{Während einer Lernsitzung}
Jetzt zur Lernsitzung an sich. Wenn eine Lernsitzung gestartet wurde, so wird die verbleibende Zeit entsprechend auf dem Bildschirm angezeigt (und auch als Benachrichtigung):
\begin{center}
    \mobile{learn_noti.png}\qquad
    \mobile{learn_screen.png}
\end{center}
Mit dem \button{\,~{\tiny\faStepForward}~\,}-Button links unten kann man eine Phase frühzeitig abschließen, mit dem \button{\,~{\tiny\faTimesCircle}~\,}-Button rechts unten kann man die Lernsitzung abbrechen.

\smash{\sidefig[survey]{\mobile{learn_tosurvey.png}}{Die Umfrage-Benachrichtigung.}}
\subsection{Das Ende einer Lernsitzung}
Zuletzt wird eine Benachrichtigung am Ende einer Lernsitzung geschickt, die darum bittet den Fragebogen auszufüllen(siehe \autoref{survey}). Dieser sollte insgesamt zumindest einmal gemacht werden, am besten einfach ab und zu nach der Lernsitzung oder wenn sich mal Zeit findet.

\clearpage
\section{Fragebogen}
Um auf den Fragebogen zu kommen, musst du oben links auf das Menü~(\faNavicon) klicken (oder von links nach rechts wischen). Dann öffnet sich folgendes Menü, bei dem auf \say{Fragebogen} geklickt werden soll:
\begin{center}
    \mobile{start_reg.png}\quad\tikz{\draw[-Kite,rounded corners=1pt] (0,0) -- (.5,0) |- (1,.75);}\quad\mobile{nav_survey.png}
\end{center}
Jetzt befindest du dich am Anfangsbildschirm des Fragebogens (\autoref{start-survey}). Bei einem Klick auf dem Bildschirm startet dann der Fragebogen.
\sidefig[start-survey]{\mobile{survey_start.png}}{Startbildschirm des Fragebogens.}%
Dort sind einige Fragen, die du dann ausfüllen sollst. Mit Wischen oder mit Klicken auf die oberen Reiter, kann man entsprechend die Fragen wechseln:
\begin{center}
    \mobile{survey_question.png}\quad\tikz{\draw[-Kite,rounded corners=1pt,densely dashed] (0,0) -- (.5,0) |- (1,.75);}\quad\mobile{survey_end.png}
\end{center}
\noindent Am Ende kannst du dann deine Antworten mit einem Klick auf den Bildschirm einreichen.

\clearpage
\section{Statistiken}
Um auf die Statistiken zu kommen, musst du oben links auf das Menü~(\faNavicon) klicken (oder von links nach rechts wischen). Dann öffnet sich folgendes Menü, bei dem auf \say{Statistiken} geklickt werden soll:
\begin{center}
    \mobile{options.png}\quad\tikz{\draw[-Kite,rounded corners=1pt] (0,0) -- (.5,0) |- (1,.75);}\quad\mobile{nav_stats.png}
\end{center}

\sidefig[stats-open]{\mobile{stats_start.png}}{Anfangs\-bild\-schirm der Statistiken.}
Beim Anfangsbildschirm (\autoref{stats-open}) kannst du zwischen zwei Typen von Statistiken auswählen. Oben sind die Fragebogenstatistiken und Unten die Lernsitzungsstatistiken.

\subsection{Die Fragebogenstatistiken}
Die Statistiken für den Fragebogen sehen so aus:
\begin{center}
    \mobile{stats_survey.png}
\end{center}

\smash{\sidefig[learn-stat-overview]{\vspace*{-1.75cm}\mobile{stats_selectionLearn.png}}{Die Lern\-sit\-zungs\-statistiken.\vspace*{.5em}}}
\subsection{Die Lernsitzungsstatistiken}
Der Anfang der Lernsitzungsstatistiken zeigt eine Auswahl der verschiedenen Lernsitzungen (anhand des Datums und des Endes der Sitzung) und auch eine gesamte Statistik (\autoref{learn-stat-overview}).
\sidefig{\mobile{stats_complete_1.png}\medskip\\
\mobile{stats_complete_2.png}}{Beispiele für die Gesamtübersicht.}Es folgen einige Beispielbilder.\vfill

\begin{center}
    Eine Lernsitzung:\medskip\\
    \mobile{stats_session_1.png}\qquad
    \mobile{stats_session_2.png}
\end{center}\vfill

\begin{center}
    Eine Lernphase:\medskip\\
    \mobile{stats_phase_1.png}\qquad
    \mobile{stats_phase_2.png}
\end{center}\vfill
\end{document}
